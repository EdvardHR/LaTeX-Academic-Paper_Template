\pagebreak
\section{Om LaTeX}

\textbf{NB! input for denne filen må fjernes fra main.tex endelig versjon kompileres og lagres!}

Litt nyttig informasjon om LaTeX:

Gjør det til en god vane å kopiere denne filen FØR du begynner å skrive i den. Da kan du trygt slette denne filen her, slik at eksempelsiteringen nedenfor ikke blir med i det endelige produktet, men at du har denne som guide til senere \footnote{Leser du denne i pdf-vinduet vil jeg anbefale deg å dobbeltklikke på et av ordene her som ikke er linker eller referanser. Da vil overleaf ta deg akkurat dit i filsamlingen teksten er skrevet og formatert i.}. 
I source-fila "Om Latex.tex" har jeg forsøkt å få frem en del nyttige ekempler.

Finner du nye pakker du liker å ha i LaTeX-biblioteket? Legg de først til i kopien du jobber i, og hvis de fungerer som de skal, kan du legge dem til i main.tex her også. Hvis du vil fjerne pakker, anbefaler jeg at du lager ei kopifil slik at du kan se at det ikke fjernes noen pakker som er avhengige av hevrandre med et uhell. 

Begynn alltid med å bestemme deg for referansestil hvis du skal bruke referasner. Det er \textbf{IKKE} \textit{krise} om du glemmer dette, eller ombestemmer deg, men det er god vane å gjøre dette med én eneste gang, slik at referansestilen passer produktet. Dette gjør du øverst i main.tex i ved å endre "style = \textit{din stil}" for pakken \textit{biblatex}.

Jeg har satt referansestilen til IEEE i utgangspunktet, da dette er referansestilen jeg foretrekker som regel. Andre nyttige referansestiler er vist i \ref{lst:referansestiler}. Andre stiler med forklaringer og bruksområder finnes
\hyperlink{https://www.reed.edu/cis/help/LaTeX/bibtexstyles.html}{her}. 
\\ %& \\ betyr bare linjeskift.
\begin{table}[!h] 
    \centering
    \caption{Noen andre referansestiler}
    \begin{tabular}{cccc}
        \hline
        plain.bst & unsrt.bst & apa.bst & alpha.bst\\
        \end{tabular}
    \label{lst:referansestiler}
\end{table}
\\
Det FØRSTE ELEMENTET (f.eks "snl\_metall") er referansenavnet 
\textbf{\textit{du}} setter for å enklere kunne sitere effektivt. 
Det går helt fint an å ha elementer i "biblioteket" du ikke bruker i tkesten, og det er kun de du benytter som vil dukke opp i referanselista. Formen vil automatisk stemme med hva slags referansestil som er valgt. Anbefaler å bruke tid på å lage gode, deskriptive navn; gjør lite om de er lange da LaTeX har svært god "autofill" ved bruk av de ulike kryssrefferanseverktøyene 
(f.eks \cite{snl_metall} for den øverste artikkelen i \textit{references.bib}).
